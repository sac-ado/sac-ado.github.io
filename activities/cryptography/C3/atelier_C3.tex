\documentclass[a4paper]{article}
 
\usepackage{lipsum}
\usepackage[a4paper, total={7in, 9in}]{geometry}
\usepackage{tikz}
\usetikzlibrary{shapes.geometric}
\usepackage[explicit]{titlesec}
\usepackage{array}
\usepackage{tabularx}

\titleformat{\section}
  {\normalfont\LARGE\bfseries}
  {}
  {0em}
  {{\titlerule[2pt]}\\[0.1em]#1}
  [\vspace{0.2em}{\titlerule[2pt]\vspace{1em}}]

\titleformat{\subsection}
  {\normalfont\Large\bfseries}
  {}%{\thesubsection}
  {0em}%{1em}
  {#1}
  [{\titlerule[1.5pt]}\vspace{0.5em}]

\titleformat{\subsubsection}
  {\normalfont\bfseries}
  {}%{\thesubsection}
  {0em}%{1em}
  {#1}

\renewcommand{\familydefault}{\sfdefault}

\newcommand\score[2]{%
  \pgfmathsetmacro\pgfxa{#1 + 1}%
  \tikzstyle{scorestars}=[star, star points=5, star point ratio=2.25, draw, inner sep=1.3pt, anchor=outer point 3]%
  \begin{tikzpicture}[baseline]
    \foreach \i in {1, ..., #2} {
      \pgfmathparse{\i<=#1 ? "yellow" : "gray"}
      \edef\starcolor{\pgfmathresult}
      \draw (\i*1.75ex, 0) node[name=star\i, scorestars, fill=\starcolor]  {};
   }
  \end{tikzpicture}%
}

\setlength\parindent{0pt}
\pagenumbering{gobble}

\begin{document}
 
\section{Atelier MP2 \hfill Mot de passe 2}

{
\setlength\arrayrulewidth{2pt}

\begin{minipage}{.55\textwidth}
  \begin{tabularx}{\textwidth}{|l|X|}
    \hline
    Durée                    & 20 minutes \\
    Organisation des enfants & {\bf Groupes}, Individuel, Collectif \\
    Nombre d'enfants         & Groupes de 20 \\
    Niveau visé              & CM1 à CM2 \\
    \hline
  \end{tabularx}
\end{minipage}
\begin{minipage}{.45\textwidth}
  \begin{tabularx}{\textwidth}{|l|X|}
    \hline
    Ambiance                       & {\bf Calme}, Actif \\
    Difficulté pour les enfants    & \score{4}{5} \\
    Complexité de mise en \oe uvre & \score{4}{5} \\
    Coût de mise en \oe uvre       & 30 euros \\
    \hline
  \end{tabularx}
\end{minipage}
}

\subsection{Liens avec les autres ateliers}

S'appuie sur aucun autre ateliers

Permet d'enchaîner avec les ateliers de social-engineering

\subsection{Objectif de l'atelier}
Cet atelier a pour but de faire comprendre que, lors de l'échange de messages, on ne peut pas toujours faire confiance à celui ou celle qui vehicule ce message. Il faut donc dans certains cas, avoir recours à des méthodes de protection des données echangés. (échanges confidentiels)

\subsection{Matériel nécessaire}
Une paire de petit cadenas et une boîte par petits groupes de réflexion (3/4 élèves)

\subsection{Principe de l'atelier}
Un message secret doit être envoyé, mais on ne fait pas confiance à la personne qui le transporte. On veut alors le verrouiller dans une boîte sans donner la clef au transporteur.
\\
Méthode : 
L'élève A essaie de faire parvenir un message secret à l'élève B et possède une boîte opaque un message (sur papier) et un cadenas a clefs L'élève B a lui juste un cadenas et une clef.
Puis l'élève C qui est le transporteur est le seul à pouvoir faire passer la boîte de l'élève A à l'élève B ou inversement. Le but est de faire passer le message sans que l'élève transporteur puisse ouvrir la boîte et lire le message secret
\\
Solution:
L'élève A ferme la boîte avec son cadenas et l'envoie a l'élève B. L'élève B ferme une deuxième fois la boîte avec son cadenas et le refait passer a l'élève A. 
L'élève A enlève son cadenas à l'aide de sa clef et le refait passer à l'élève B. Enfin l'élève B enlève son propre cadenas et peut ouvrir la boîte et lire le message.


\subsection{Déroulement}
\begin{itemize}
    \setlength\itemsep{0em}
    \item Fournir deux cadenas et une boîte opaque à chaque groupe de réflexion.
    \item Expliquer la problématique de confiance autour de l'élève transporteur.
    \item Laisser les élèves réfléchir autour de cet atelier.
    \item Expliquer la solution aux élèves.
    \item Faire comprendre l'intérêt des solutions de sécurisation des échanges de messages dans les systems modernes (médium non fiable/publique)
\end{itemize}

\subsubsection{Avant la séance}
\begin{itemize}
  \setlength\itemsep{0em}
  \item Imprimer la fiche
  \item Amener le matériel dans la salle
  
\end{itemize}

\subsubsection{Durant la séance}
\begin{itemize}
  \setlength\itemsep{0em}
  \item Poser les questions d'entrée aux enfants et noter les réponses
  \item Présenter le contexte (cf. section Objectif de l'atelier)
  \item Discuter avec les élèves de la confiance qu'ils portent dans les application d'échange de messages aujourd'hui
  \item Discuter des différents moyens qu'il existe pour envoyer et recevoir des messages et de leurs apparente sécurisation (lettres, mails, sms, appels, applications, voix orale, etc...)
\end{itemize}

\subsubsection{Après la séance}
\begin{itemize}
  \setlength\itemsep{0em}
  \item Ranger le matériel
  \item Demander aux enfants ce qu'ils retiennent de cette expérience (le but étant d'ouvrir à la réflexion de l'enfant et à la critique)
\end{itemize}

\subsection{Questions d'entrée à poser en début d'atelier}
TODO
\subsection{Questions de fin à poser en fin d'atelier}
TODO
Question + réponse attendue : remarque à faire si trop de réponses éloignées

\subsection{Message à faire passer à la fin}
TODO
\end{document}

