\documentclass[a4paper]{article}
 
\usepackage{lipsum}
\usepackage[a4paper, total={7in, 9in}]{geometry}
\usepackage{tikz}
\usetikzlibrary{shapes.geometric}
\usepackage[explicit]{titlesec}
\usepackage{array}
\usepackage{tabularx}

\titleformat{\section}
  {\normalfont\LARGE\bfseries}
  {}
  {0em}
  {{\titlerule[2pt]}\\[0.1em]#1}
  [\vspace{0.2em}{\titlerule[2pt]\vspace{1em}}]

\titleformat{\subsection}
  {\normalfont\Large\bfseries}
  {}%{\thesubsection}
  {0em}%{1em}
  {#1}
  [{\titlerule[1.5pt]}\vspace{0.5em}]

\titleformat{\subsubsection}
  {\normalfont\bfseries}
  {}%{\thesubsection}
  {0em}%{1em}
  {#1}

\renewcommand{\familydefault}{\sfdefault}

\newcommand\score[2]{%
  \pgfmathsetmacro\pgfxa{#1 + 1}%
  \tikzstyle{scorestars}=[star, star points=5, star point ratio=2.25, draw, inner sep=1.3pt, anchor=outer point 3]%
  \begin{tikzpicture}[baseline]
    \foreach \i in {1, ..., #2} {
      \pgfmathparse{\i<=#1 ? "yellow" : "gray"}
      \edef\starcolor{\pgfmathresult}
      \draw (\i*1.75ex, 0) node[name=star\i, scorestars, fill=\starcolor]  {};
   }
  \end{tikzpicture}%
}

\setlength\parindent{0pt}
\pagenumbering{gobble}

\begin{document}
 
\section{Atelier S2 \hfill Social Engineering 2}

{
\setlength\arrayrulewidth{2pt}

\begin{minipage}{.55\textwidth}
  \begin{tabularx}{\textwidth}{|l|X|}
    \hline
    Durée                    & 20 minutes \\
    Organisation des enfants & {\bf Collectif}, Individuel, Groupes \\
    Nombre d'enfants         & Groupes de 20 \\
    Niveau visé              & CM1 à CM2 \\
    \hline
  \end{tabularx}
\end{minipage}
\begin{minipage}{.45\textwidth}
  \begin{tabularx}{\textwidth}{|l|X|}
    \hline
    Ambiance                       & {\bf Actif} \\
    Difficulté pour les enfants    & \score{2}{5} \\
    Complexité de mise en \oe uvre & \score{2}{5} \\
    Coût de mise en \oe uvre       & 0 euro \\
    \hline
  \end{tabularx}
\end{minipage}
}

\subsection{Liens avec les autres ateliers}
 
Cet atelier couvre les différents thèmes abordés (vie privée, mots de passe, cryptographie et sténographie) sous forme de synthèse.

\subsection{Objectif de l'atelier}










\subsection{Matériel nécessaire}

Pour mettre en place cet atelier, il faut un ordinateur et un vidéo-projecteur.

\subsection{Principe de l'atelier}

\subsection{Déroulement}

\subsubsection{Avant la séance}

\subsubsection{Durant la séance}

\subsubsection{Après la séance}

\subsection{Questions d'entrée à poser en début d'atelier}

\subsection{Questions à poser lors de la discussion}

\subsection{Questions de fin à poser en fin d'atelier}

\subsection{Message à faire passer à la fin}

\subsection{Script}

    \subsubsection{Scène 1}
    
        \textit{Camille va à l'école, une séance dans la salle informatique est prévue, pendant laquelle elle consultera ses mails.}
        
        Dans un premier temps, Camille doit saisir ses identifiants. Une fois saisis, le navigateur propose de sauvegarder automatiquement ces derniers pour lui en faciliter l'usage. \\
        
        CHOIX :
        \begin{itemize}
            \item Oui -- game over (expliquer pq)
            \item Non -- non, on continue
        \end{itemize}
        A COMPLETER
    
    \subsubsection{Scène 2}
    
        \textit{Lorsque Camille consulte ses mails, un mail en particulier attire son attention dont l'objet est "VOUS AVEZ GAGNE UNE TABLETTE !!!"} \\
        
        Camille ouvre le mail correspondant et lit : "Cliquez sur le lien pour recevoir sur votre tablette gratuitement !".
        
        CHOIX :
            \begin{itemize}
                \item Cliquer sur le lien -- pas bien
                \item Agir -- mettre l'expéditeur en indésirable
            \end{itemize}
            A COMPLETER
        
    \subsubsection{Scène 3}
    
        \textit{Une fois que Camille a consulté ses mails, elle doit laisser allumer l'ordinateur pour son camarade. } \\
        
        A quoi doit-elle penser ?
        \begin{itemize}
            \item Partir -- pas bien
            \item Se déconnecter de son compte mail -- explications
        \end{itemize}
            A COMPLETER
            
    \subsubsection{Scène 4}
    
        \textit{En quittant la salle, Camille voit une clé usb par terre et la ramasse.} \\
        
        Que fait-elle ?
        \begin{itemize}
            \item Elle la branche sur son pc pour voir ce qui est dedans et retrouver son propriétaire -- pas bien
            \item Elle la donne à l'enseignant --  safe
        \end{itemize}
            A COMPLETER
        
    \subsubsection{Scène 5}
        \textit{Après l'école, Camille rentre chez elle et retrouve son chien Pedro. Elle décide ensuite de jouer à un nouveau jeu vidéo en ligne que lui ont conseillée ses amis. Pour cela, elle doit créer un compte et donc un mot de passe.} \\
        
        Quel mot de passe choisit-elle ?
        \begin{itemize}
            \item PeDRo
            \item 123456
            \item Bleu82Paris63
            \item camille
        \end{itemize}
            A COMPLETER
            
    \subsubsection{Scène 6}
        \textit{Après avoir choisi son mot de passe, Camille décide de le noter quelque part pour s'en souvenir.} \\
        
        Où écrit-elle le mot de passe ?
        \begin{itemize}
            \item sur un post-it qu'elle colle sous le clavier
            \item dans son téléphone
            \item dans son cahier de mathématiques
            \item dans son carnet secret qui est caché et verrouillé avec un petit cadenas
        \end{itemize}
            A COMPLETER
            
    \subsubsection{Scène 7}
        \textit{Camille peut enfin jouer au jeu vidéo et lance une partie. Peu de temps après, lolo\_du\_31 souhaite discuter avec elle. Elle ne connaît pas cette personne.} \\
        
        Que fait-elle ?
        \begin{itemize}
            \item Elle ignore le message et continue à jouer
            \item Elle répond à lolo\_du\_31
        \end{itemize}
            A COMPLETER
            
    \subsubsection{Scène 8}
    
    SCENE POUR CONCLURE ?

\end{document}
